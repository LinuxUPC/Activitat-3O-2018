\documentclass{article}
\usepackage[utf8]{inputenc}
\usepackage{hyperref}

\title{Activitat LinuxUPC per el 3-O}
\author{LinuxUPC}
\date{2018-10-03}

\begin{document}
\maketitle

\section{Introducció}
\p{En aquesta activitat us trobareu un codi en C on haureu d'anar solucionant errors per fer que continuï la seva execució fins al final.
Quan mes lluny arribeu, mes punts guanyareu}
\section{Us de l'activitat}
\p{Aquesta activitat esta pensada per a usuaris amb nocions bàsiques de entorns de treball, per tant em facilitat la feina de compilar. Amb la tecla \textit{F5} compilareu el codi i l'executareu, preneu la tecla \textit{ENTER} per a tornar a l'editor i la tecla \textit{i} per editar el codi.\newline Amb la tecla \textit{F6} el codi tornara al estat inicial.}
\section{Puntuació}
\p{La puntuació anirà del 0 al 10, en el annex del codi es pot veure exactament quants punts dona cada cosa. \newline Mencionar que el ultim punt es deixa als participants ser creatius i impressionar als jurats.}
\section{Annex}
\p{El codi utilitzat per la prova es pot trobar al següent enllaç: \newline
\url{https://github.com/johanitus1/ActivitatTeresa}

}
\end{document}
